\documentclass[a4paper]{article}


%opening
\title{Notes on $HD^+$ Propagation}
\author{Tore Birkeland}

\begin{document}

\maketitle

\section{The Problem}
We wish to simulate certain diatomic ionic molecules ($HD^+$, $H_2^+$, $D_2^+$) under a femtosecond laser pulse. 

\subsection{Center of mass Coordinates}
A diatomic molecule has the following hamiltonian in the laboaratory frame in atomic units
\begin{equation}
 H = - \frac{1}{2 M_1} \nabla^2_{R_1} - \frac{1}{2 M_2} \nabla^2_{R_2} - \frac{1}{2 m_e} \nabla^2_{r_e} + \frac{Z_1 Z_2}{|R_1 - R_2|} - \frac{Z_1}{|r_e - R_1|} - \frac{Z_2}{|r_e - R_2|}
\end{equation}
Which can be transformed into center of mass coordinates by the transformation
\begin{eqnarray}
	R_c &=& \frac{M_1 R_1  + M_2 R_2 + m_e r_e}{M_c} \\
	R &=& R_1 - R_2 \\
	r &=& r_e - \frac{M_1 R_1 + M_2 R_2}{M_1 + M_2}
\end{eqnarray}
The Hamiltonian in the new coordinates is then
\begin{equation}
\label{eqn:full-hamiltonian-cm}
 H = - \frac{1}{2 M_c} \nabla^2_{R_c} - \frac{1}{2 M_n} \nabla^2_{R_n} - \frac{1}{2 \mu} \nabla^2_{r} + \frac{Z_1 Z_2}{|R|} - \frac{Z_1}{|r - \rho_1 R|} - \frac{Z_2}{|r - \rho_2 R|}
\end{equation}
where
\begin{eqnarray*}
 M_c &=& M_1 + M_2 + m_e \\
 M_n &=& \frac{M_1 M_2}{M_1 + M_2} \\
 \mu &=& \frac{m_e(M_1 + M_2)}{M_c} \\
 \rho_i &=& \frac{M_i}{M_1 + M_2} 
\end{eqnarray*}

\subsection{Born Openheimer Approximation}
We will neglect the effect of the laser on the center-of-mass system, and therefore the first term in \ref{eqn:full-hamiltonian-cm}, will separate out, and the system is reduced from a nine to six dimensions. Furthermore, we will assume the Born-Openheimer approximation, in which one assumes that movement in the $r$ direction will be instantaneous compared to movement in the $R$ direction, and one can therefore separate the equation into two three dimensional equations.
\begin{equation}
	\Psi(R, r) = \psi(R) \phi(r; R)
\end{equation}
\begin{equation}
	H(R, r) = H_n(R) + H_e(r; R)
\end{equation}
Where
\begin{eqnarray}
	H_n(R) &=& - \frac{1}{2 M_n} \nabla^2_{R_n} + \frac{Z_1 Z_2}{|R|} \\
	H_e(r; R) &=&  - \frac{1}{2 \mu} \nabla^2_{r} - \frac{Z_1}{|r - \rho_1 R|} - \frac{Z_2}{|r - \rho_2 R|}
\end{eqnarray}

One first solves the electronic system paramterically for R
\begin{equation}
	\label{eqn:electron-hamiltonian-bo}
	H_e(r; R) \phi_i(r; R) = \epsilon_i(R) \phi_i(r; R)
\end{equation}
This gives a set of energy curves $\epsilon(R)$, which can be used to solve the the nuclear system
for each electronic state $i$
\begin{equation}
	\label{eqn:nuclear-hamiltonian-bo}
 	H_n^i(R) \psi_{i,j}(R) = \left( H_n(R) + \epsilon_i(R) \right) \psi_{i,j}(R) = E_{i,j} \psi_{i,j}(R)
\end{equation}


\subsection{Laser Interaction}
We will model the laser field with the usual semi classical model. For a polarized laser pulse in 
the length gauge, this is in the laboratory frame
\begin{equation}
 H_l = e E(t) \varepsilon ( r - Z_1 R_1 - Z_2 R_2)
\end{equation}
Which in the center of mass frame is (with center of mass motion disregarded)
\begin{equation}
 H_l = E(t) \varepsilon ( \kappa_r r - \kappa_R R )
\end{equation}
\begin{eqnarray}
	\kappa_r &=& \frac{(Z_1 + Z_2) m_e + (M_1 + M_2)}{M_c} \\
	\kappa_R &=& \frac{Z_1 M_1 - Z_2 M_2}{M_1 + M_2]}
\end{eqnarray}
If the electronic system in the Born-Openheimer approximation is solved $\{ \phi_i(r; R) \}$, one can calculate the induced dipole moment 
between two electronic states $\phi_i(r; R)$ and $\phi_j(t; R)$
\begin{equation}
	d_{i,j}(R) = \int \phi_i^*(r; R) r \phi_j(r, R) dr
\end{equation}
If we put this and eqn \ref{eqn:nuclear-hamiltonian-bo} into the time dependent Schrödinger equation, we get
\begin{equation}
 - \imath \frac{\partial}{\partial t} \psi_i = \left( H_n + \epsilon_i(R) \right) \psi_i + E(t) \varepsilon \left( \kappa_r \sum_j d_{i, j}(R) \psi_j + \kappa_R R \psi_i\right)
\end{equation}
Usually, however, only the two lowest electronic states have to be considered, in which case this simplifies to two coupled equations.


\section{Numerical Method}
We will use a standard splitting technique for time marching the TDSE. The Hamiltonian is split in three parts
\begin{eqnarray}
	H_1 &=& \frac{1}{2 M_n} \nabla^2_R \\
	H_2 &=& \frac{Z_1 Z_2}{|R|} + \epsilon_i(R) \\
	H_3 &=& E(t) \varepsilon \left( \kappa_r \sum_j d_{i, j}(R) + \kappa_R R \right)
\end{eqnarray}
\begin{equation}
	- \imath \frac{\partial}{\partial t} \mathbf{\psi}(t) = (H_1 + H_2 + H_3) \mathbf{\psi}(t)
\end{equation}
A truncated Magnus expansion of the solution can be written as
\begin{equation}
 	\psi(t + \Delta t) = \exp( - \imath \Delta t (H_1 + H_2 + H_3) ) \psi(t) + O(\Delta t^2)
\end{equation}
Since each of the sub-Hamiltonians can be solved accurately (see below), it makes sense to write the solution in terms of the propagators for the individual sub-Hamiltonians
\begin{equation}
	\psi(t + \Delta t) = \exp(- \imath \Delta t H_1) \exp(- \imath \Delta t H_2) \exp(- \imath \Delta t H_3) \psi(t) + O(\Delta t^2)
\end{equation}
The accuracy of this splitting can be increased by considering other splitting schemes such as the symmetric Strang splitting (REF)

\subsection{Solving $H_1$}
\subsection{Solving $H_2$}
\subsection{Solving $H_3$}
\end{document}
